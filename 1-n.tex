
\chapter{Prerequisite}

\section{Important Things in a Math Book}

\chapter{Limits and Continuity}

\section{Limit of a Function}

\section{Limit of a Sequence}
\subsection{Convergence and Divergence}

\section{Limit Laws of Function and Sequence}
\subsection{Limit Laws of Function}
\subsection{Limit Laws of Sequence}
\subsection{The Sandwich Theorem}
\subsection{An Important Limit}

\section{Continuity}
\subsection{}

\chapter{Derivatives}
\section{Big-O and Little-o Notation}
\textbf{Big-O Notation}, along with \textbf{Little-o Notation}, tell you how fast a function grows and declines.
Obviously, this notation is abusing the equality symbol, since it violates the axiom of
equality: "things equal to the same thing are equal to each other"
. To be more formally
correct, some people (mostly mathematicians, as opposed to computer scientists) prefer
to define O(g(x)) as a set-valued function, whose value is all functions that do not grow
faster then g(x), and use set membership notation to indicate that a specific function is a
member of the set thus defined. Both forms are in common use, but the sloppier equality
notation is more common at present.

\chapter{Applications of Derivatives}
\section{Motivation}
% Creating logic sequence with arrows for theorems
\[
\begin{tikzpicture}[>=Stealth, thick]
    % Intermediate Value Theorem
    \node (IVT) at (0, 0) {Intermediate Value Theorem};
    % Extreme Value Theorem
    \node (EVT) at (0, -2) {Extreme Value Theorem};
    % Rolle's Theorem
    \node (RT) at (0, -4) {Rolle's Theorem};
    % Mean Value Theorem
    \node (MVT) at (0, -6) {Mean Value Theorem};
    % Cauchy's Mean Value Theorem
    \node (CMVT) at (0, -8) {Cauchy's Mean Value Theorem};
    
    % Arrows showing implication
    \draw[->] (IVT) -- (EVT);
    \draw[->] (EVT) -- (RT);
    \draw[->] (RT) -- (MVT);
    \draw[->] (MVT) -- (CMVT);
\end{tikzpicture}
\]


\section{Theorems and Tools for Analyzing Function Extremes}
\subsection{Intermediate Value Theorem}
\subsection{Extreme Value Theorem}
\subsection{Rolle's Theorem}
\subsection{Mean Value Theorem}

\section{Indeterminate Forms and L'Hôpital's Rule}
\subsection{Cauchy's Mean Value Theorem}
\subsection{Indeterminate Form $0/0$}
\subsection{L'Hôpital's Rule}

\chapter{Integrals}
\section{Motivation}
\subsection{Definite and Indefinite Integrals}

\section{Antiderivatives}

\section{Riemann Sums}


\section{The Fundamental Theorem of Calculus}

\chapter{Techniques of Integration}
\section{Substitution}


\section{Integration by Parts}



\section{Partial Fractions}

\chapter{Applications of Definite Integrals}

\chapter{}
